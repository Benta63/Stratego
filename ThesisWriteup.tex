% !TEX TS-program = pdflatex
% !TEX encoding = UTF-8 Unicode

% This is a simple template for a LaTeX document using the "article" class.
% See "book", "report", "letter" for other types of document.

\documentclass[12pt]{article} % use larger type; default would be 10pt

\usepackage[utf8]{inputenc} % set input encoding (not needed with XeLaTeX)

%%% Examples of Article customizations
% These packages are optional, depending whether you want the features they provide.
% See the LaTeX Companion or other references for full information.

%%% PAGE DIMENSIONS
\usepackage{geometry} % to change the page dimensions
\geometry{a4paper} % or letterpaper (US) or a5paper or....
% \geometry{margin=2in} % for example, change the margins to 2 inches all round
% \geometry{landscape} % set up the page for landscape
%   read geometry.pdf for detailed page layout information

\usepackage{graphicx} % support the \includegraphics command and options

% \usepackage[parfill]{parskip} % Activate to begin paragraphs with an empty line rather than an indent

%%% PACKAGES
\usepackage{booktabs} % for much better looking tables
\usepackage{array} % for better arrays (eg matrices) in maths
\usepackage{paralist} % very flexible & customisable lists (eg. enumerate/itemize, etc.)
\usepackage{verbatim} % adds environment for commenting out blocks of text & for better verbatim
\usepackage{subfig} % make it possible to include more than one captioned figure/table in a single float
\usepackage{biblatex} %For the bibliography
\addbibresource{sample.bib}

%%% HEADERS & FOOTERS
\usepackage{fancyhdr} % This should be set AFTER setting up the page geometry
\fancyhf{} %Clearing
\fancyhead[R]{<Stolz \thepage>} %Header
\pagestyle{fancy} % options: empty , plain , fancy
\renewcommand{\headrulewidth}{0pt} % customise the layout...


%%% SECTION TITLE APPEARANCE
\usepackage{sectsty}
\allsectionsfont{\sffamily\mdseries\upshape} % (See the fntguide.pdf for font help)
% (This matches ConTeXt defaults)

%%% ToC (table of contents) APPEARANCE
\usepackage[nottoc,notlof,notlot]{tocbibind} % Put the bibliography in the ToC
\usepackage[titles,subfigure]{tocloft} % Alter the style of the Table of Contents
\renewcommand{\cftsecfont}{\rmfamily\mdseries\upshape}
\renewcommand{\cftsecpagefont}{\rmfamily\mdseries\upshape} % No bold!

%%% END Article customizations

%%% The "real" document content comes below...

\title{Cognitive Science Thesis}
\usepackage{authblk}
\author[1]{Noah Stolz}
\affil[1]{Rensselaer Polytechnic Institute}
\date{}
%\date{} % Activate to display a given date or no date (if empty),
         % otherwise the current date is printed

\begin{document}
\maketitle

\section{Introduction}

Stuf

\subsection{A subsection}

More text.
\section{Prior Work in Artificial Intelligence and Games}
\t Many games have been solved using Artificial Intelligence, some even with neural networks. %Cite this%
However, many of these games have perfect information %AlphaGo (cite), that checkers ting, Deep Blue.
Perfect information means that all the information is visible, however, not all games have perfect information. In some games, primarily card games, the player does not know everything. For example in Bridge, the player does not know the opposing players hand. This is the fundamental difficulty in solving imperfect information games. It is not only the fact that the setup can be different, but also the fact that predicting your opponent's moves are much harder given that we do not know the opposing player's hands or pieces. 
\section{Stratego}
\t Stratego is also an imperfect information game, however, with a different goal that card games such as bridge or poker. The goal of Stratego is to capture the opponent's flag which is the 'F' on figure %What figure??
%Reference the game you have (Or Hasbro)
To do this, we move the numbered pieces around on the green spaces of the board in figure %What figure
. Each of the numbered pieces can move one square with the exception of the '2', also known as the Scout which can move any distance in any direction. If one piece runs into a piece of the other color, both pieces are revealed and the higher numbered piece goes on to occupy the square. There are few exceptions to this rule. One such exception is that the Spy, denoted by the 'S', which normally moves like a numbered piece, can win a battle only if it attacks a '10', or general. Another exception is that if the two pieces equal, both are removed. Additionally, there is another piece denoted by the 'B' which stands for a bomb. The bomb cannot move, however, if any piece, except the '3', attacks the bomb, that piece is removed. The '3', or miner, defuses the bomb and thus removes the bomb from the board. 
\t 
%Picture of Stratego board
\section{Results}
\section{Discussion}
\section{Conclusions}
\section{Future Work}

\section{Acknowledgements}

%%Have a bibliography




\end{document}
